\chapter{Implementación en entorno virtualizado} % sistema completo seL4 en QEMU
\section{Diseño del experimento}
\begin{itemize}
    \item Se realizará en completamente en host. ¿Por qué? Simplicidad de implementación.
\end{itemize}
\section{Procedimiento} % Proceso de: linux_QEMU-linux_QEMU -> 3VMsel4_QEMU. 3VMsel4_QEMU
  
\subsection{Construcción de un kernel Linux con soporte para WireGuard}
Actualmente, el VMM de seL4 implementado en CAmkES para la arquitectura x86 cuenta con soporte probado para el kernel Linux 4.9. En este trabajo se utilizó una versión limpia de este kernel, obtenida de la rama \textit{stable} de Linux 4.9.y, en particular la versión 4.9.337, que se encuentra disponible en el repositorio oficial de Linux. Al momento de configurar la compilación del kernel, se tuvo en cuenta la compatibilidad con el hipervisor seL4, cuyos requisitos se encuentran documentados en la página oficial del proyecto \textit{camkes-vm} para seL4 \cite{camkes_vm}.

El soporte nativo para WireGuard fue añadido al kernel Linux en la versión 5.6, sin embargo, el repositorio \textit{wireguard-linux-compat} proporciona un parche que permite compilar WireGuard en versiones anteriores del kernel, como la utilizada \cite{wireguard-compilation}.

\begin{lstlisting}[language=bash, caption={Parche wireguard-linux-compat para kernel 4.9}]
git clone https://git.zx2c4.com/wireguard-linux-compat
./wireguard-linux-compat/kernel-tree-scripts/jury-rig.sh ./linux-stable
\end{lstlisting}

\begin{itemize}
    \item De www.wireguard.com/compilation/kernel-requirements se obtuvieron los requerimientos.
    \item wireguard-linux-compat patch
    \item seL4 soporta kernel 4.9[ref], se compiló este kernel con una adaptación de .config file original de los ejemplos camkes-vm
    \item Buscar referencias para justificar todo. \cite{Laricch2009AAMOP_ICBp3}

    \item Drivers de red compatibles con el hardware utilizado.
\end{itemize}

\subsection{Generación de una imagen de sistema mediante Buildroot}
\begin{itemize}
    \item Se utilizó la versión 2023.02.1 de Buildroot utilizando el kernel modificado.
    \item Se configuró el sistema de archivos para que contenga los binarios necesarios para el funcionamiento de WireGuard y las herramientas de red.
\end{itemize}

\subsection{Adaptación del ejemplo \textit{zmq\_samples}}
\begin{itemize}
    \item Se adaptó el ejemplo de CAmkES para que funcione con el nuevo kernel y la imagen de sistema generada. Basicamente tocar el CMakeLists.txt
    \item ZeroMQ. Problema con iperf3 solucionado aumentando tamaño de buffers.
    \item Incrementar RAM de las VMs.
\end{itemize}

\subsection{Configuración del \textit{passthrough} de interfaz Ethernet}
\begin{itemize}
    \item Interfaz e1000 QEMU. PCI. BARS. IRQ.
    \item Modificar la configuración de la VMM camkes en seL4 para permitir el acceso a los recursos PCI correspondientes.
    \item Solución al problema de utilizar dos interfaces de red. Diferentes IRQ.
    \item Funcionamiento de passthrough (colas).
\end{itemize}

\section{Integración}
\begin{itemize}
    \item Bridge en host. 4 instancias de QEMU. 2 PCs y 2 encriptadores.
\end{itemize}

\section{Validación} %verificar requerimientos
