%%%%%%%%%%%%%%%%%%%%%%%%%%%%%%%%%%%%%%%%%%%%%%%%%%%%%%%%%%%%%%%%%%%%%%%%%%%%%%%%
% \documentclass[12pt,papel,twoside]{ibtesis}
% \documentclass[12pt,papel,singlespace,oneside]{ibtesis}
% \documentclass[12pt,papel,preprint,singlespace,oneside]{ibtesis}

\documentclass[screen,pagebackref]{ibtesis}
% Antes acá estaba
% \documentclass[12pt,screen,twoside,pagebackref]{ibtesis}


%%%%%%%%%%%%%%%%%%%%% Paquetes extra %%%%%%%%%%%%%%%%%%%%%%%%%%%%%%%%%%%%%%%%%%%
% Por conveniencia: aqu\'{\i} puede cargar todos los paquetes y definir los comandos 
% que necesite
\usepackage{ibextra}
\usepackage{subcaption}


\usepackage{todonotes}
\usepackage{pdfcomment}
\usepackage{listings}
\usepackage{xcolor}  % for coloring
\renewcommand{\lstlistingname}{Código}
\renewcommand{\lstlistlistingname}{Índice de códigos}

\lstset{
  language=C,                   % Idioma del código
  basicstyle=\ttfamily\small,  % Fuente monoespaciada pequeña
  keywordstyle=\color{blue}\bfseries, % Palabras clave en azul y negrita
  commentstyle=\color{gray},   % Comentarios en gris
  stringstyle=\color{red},     % Cadenas en rojo
  numbers=left,                % Numerar líneas a la izquierda
  numberstyle=\tiny,           % Tamaño de los números de línea
  numbersep=5pt,               % Separación de los números
  backgroundcolor=\color{white}, % Fondo blanco
  frame=single,                % Marco alrededor del código
  breaklines=true,             % Romper líneas largas
  captionpos=b,                % Colocar el título debajo del código
  literate={á}{{\'a}}1 {é}{{\'e}}1 {í}{{\'i}}1 {ó}{{\'o}}1 {ú}{{\'u}}1 {ñ}{{\~n}}1 {Ñ}{{\~N}}1, % Soporte para caracteres especiales
  extendedchars=true,
  inputencoding=utf8
}


%\usepackage{hyphen-spanis}
%%%%%%%%%%%%%%%%%%%%%%%%%%%%%%%%%%%%%%%%%%%%%%%%%%%%%%%%%%%%%%%%%%%%%%%%%%%%%%%%
%%%%%%%%%%%%%%%%%%%%% Informacion sobre la tesis %%%%%%%%%%%%%%%%%%%%%%%%%%%%%%%
\title{Mi súper interesante PI}
\author{Juan Teleco}
\director{Ing. Pepe Antenil}
%\codirector{Dr.~J.~Otro m\'{a}s}
\carrera{Proyecto Integrador de la Carrera de Ingeniería en Telecomunicaciones}
\grado{Estudiante}
\laboratorio{Telecolandia}
\jurado{Señor Malo 1 (INVAP) \\ 
Señora Mala 2 (Instituto Balseiro)}
\palabrasclave{formato de Tesis, Lineamientos de escritura, Instituto Balseiro}
\keywords{Thesis format, Templates, Instituto Balseiro}
% Si queremos poner la fecha manualmente:
% \date{Diciembre de 2099}

%%%%%%%%%%%%%%%%%%%%%%%%%%%%%%%%%%%%%%%%%%%%%%%%%%%%%%%%%%%%%%%%%%%%%%%%%%%%%%%%
%\titlepagefalse % Si no quiere compilar la portada descomente esta linea
%\includeonly{apendices} % Compilar s\'{o}lo estos archivos 
\graphicspath{{figs/}} % Lugar donde encontrar las figuras generales (se puede poner uno en cada cap{\'{\i}}tulo)
%%%%%%%%%%%%%%%%%%%%%%%%%%%%%%%%%%%%%%%%%%%%%%%%%%%%%%%%%%%%%%%%%%%%%%%%%%%%%%%%


\begin{document}

% Dentro del environment 'preliminary' va:
% la dedicatoria, resumen, abstract, indices
% \begin{preliminary}

% % Escriba su dedicatoria
% \dedicatoria{
% A todos los telequitos
% }

% %%% \'{I}ndices %%%%

% \begin{abreviaturas}
%                                 %Abreviaturas
% \end{abreviaturas}

\tableofcontents                %\'{I}ndice

% \listoffigures                  %Figuras

% \listoftables                   %Tablas

% \include{Capitulos/Abstract}

% \end{preliminary}

%% acá deberían incluirse los capítulos
\chapter{Introducción}
    \section{Motivación}
    \section{Concepto de operaciones}
        \subsection{Antecedentes}
        \subsection{Suposiciones y restricciones}
        \subsection{Resumen del sistema propuesto}
        \subsection{Objetivos, metas y justificación del sistema}
        \subsection{Usuarios y modos de operación}
    \section{Requerimientos}
        \subsection{Funcionales}
        \subsection{De rendimiento}
        \subsection{De interfaz}

\chapter{Arquitectura propuesta}
\section{Descripción de tecnologías}
\subsection{WireGuard}
Ejemplo de un archivo de configuración wg0.conf
\subsection{seL4}
\subsection{CAmkES}

Trabajar con un microkernel como seL4 implica que la mayoría de las funcionalidades del sistema deben implementarse en el espacio de usuario. Para simplificar este proceso, seL4 incluye CAmkES, un \textit{framework} que facilita el desarrollo de sistemas de software modulares y seguros, diseñados específicamente para ejecutarse sobre seL4.

\textcolor{blue}{Modular refiere a que CAmkES es un modelo de componentes. Seguro refiere a. Detallar el componente VMM, será utilizado después.}

\begin{figure}[h!]
    \centering
    \includegraphics[width=0.5\textwidth]{example-image}
    \caption{VMM.}
    \label{diag:camkes_vmm}
\end{figure}

VirtQueue

Comunicación VM - VMM - seL4 - Hardware

\subsubsection{zmq\_samples}
Uno de los ejemplos de uso de VMs en CAmkES es el proyecto zmq\_samples, que implementa un sistema de comunicación entre VMs utilizando la librería de mensajería ZeroMQ. Cada VM contiene una interfaz de red virtual eth0 que se conecta a las interfaces eth0 de las demás. En la figura \ref{diag:zmq_samples} se esquematiza el funcionamiento de este sistema.



\begin{figure}[h!]
    \centering
    \includegraphics[width=0.5\textwidth]{example-image}
    \caption{Esquema de comunicación zmq\_samples.}
    \label{diag:zmq_samples}
\end{figure}



\section{Arquitectura lógica}
\subsection{Dominios}
% \subsection{Dominio A - Negro}
% \subsection{Dominio B - Encriptador}
% \subsection{Dominio C - Rojo}
    

\chapter{Estrategia de modelos en entornos virtualizados}

Con el fin de abordar de manera ordenada la complejidad del sistema y validar progresivamente cada uno de sus componentes, se adoptó una estrategia basada en la construcción de modelos incrementales. Estos modelos permiten simular, probar y verificar distintas funcionalidades antes de integrarlas en la solución final.

Los modelos descritos a continuación tienen como finalidad:
\begin{itemize}
    \item Ligar problemas concretos a cada modelo y resolverlos de forma independiente. % Tener que debuggear todos en un solo modelo puede ser muy complejo.
    \item Obtener una solución funcional en un entorno virtualizado como último paso previo a implementarla sobre \textit{hardware}.
\end{itemize}

El detalle de cada modelo tiene como objetivo proporcionar una visión clara de su propósito, las validaciones que se intentan realizar y los aspectos que quedan fuera del alcance de cada uno. Además, en cada modelo se describen aspectos relevantes sobre las tecnologías utilizadas.

\section{Comunicando sitios seguros con WireGuard}
% Descripción del modelo
El primer modelo propuesto se centra en la comunicación entre múltiples sitios seguros a través de un túnel VPN configurado mediante WireGuard. Además, se permite el acceso de los sitios a otros servicios en Internet por fuera del túnel. Se trata de un modelo de baja complejidad cuyo objetivo principal es familiarizarse con WireGuard y su configuración, así como utilizar herramientas como Wireshark para el análisis de paquetes de red. En una primera instancia, se implementó la topología de la figura \ref{diag:wg_minimal} utilizando GNS3, herramienta que permite instanciar y conectar múltiples máquinas virtuales, integrando herramientas de análisis de redes como Wireshark. 

Este modelo no pretende validar la arquitectura lógica del encriptador, sino que se enfoca en la configuración de WireGuard. Es por esto que cada encriptador se representa mediante una sola VM, en la cual se configura WireGuard. Se busca verificar que cada sitio pueda comunicarse con los demás a través del túnel VPN y acceder a Internet de manera directa.

Este modelo permite identificar y resolver problemas básicos de configuración de WireGuard, comprender el funcionamiento de las claves públicas y privadas, y analizar el tráfico cifrado y no cifrado mediante capturas en Wireshark.

\begin{figure}[h!]
    \centering
    \includegraphics[width=0.6\textwidth]{../figs/gns3_1.png}
    \caption{Otro esquema mejorcito.}
    \label{diag:wg_minimal}
\end{figure}

En una siguiente iteración de este modelo se descartó la utilización de GNS3 y se optó por instanciar cada VM mediante QEMU % cosa que GNS3 ya lo hace en backend...
, adquiriendo mayor control sobre los dispositivos de red emulados. Un ejemplo de esto es la posibilidad de observar los parámetros PCI de las interfaces de red, lo cual es relevante para la implementación del \textit{passthrough} de las mismas en la solución final.

En este modelo, además, se valida el correcto funcionamiento tanto del kernel de Linux como del \textit{filesystem} inicial, modificados para incluir soporte para WireGuard. Estas imágenes serán posteriormente utilizadas en cada VMM gestionada por seL4.

\section{Introduciendo la arquitectura lógica del encriptador}
Una vez planteada la arquitectura lógica del sistema, se procedió con su simulación, realizada utilizando GNS3. Este modelo permite limitar la complejidad de implementar la arquitectura a configurar la interconexión de las VMs que conforman el encriptador. Se tiene como objetivo validar la arquitectura lógica propuesta y verificar las funcionalidades de red pretendidas para el encriptador como lo es el \textit{split-tunneling}.

En la figura \ref{diag:gns3_2} se muestra la topología del modelo en GNS3 para simular el sistema completo. Aquí el encriptador se implementa como una interconexión mediante interfaces de red de tres VMs independientes. 

\begin{figure}[h!]
    \centering
    \includegraphics[width=0.8\textwidth]{../figs/gns3_2.png}
    \caption{Otro esquema mejorcito.}
    \label{diag:gns3_2}
\end{figure}

Como \textit{output} de este modelo se obtienen las configuraciones de red necesarias (tablas de enrutamiento y reglas de firewall) para darle al encriptador la funcionalidad pretendida. Estas configuraciones se utilizarán posteriormente en la solución.

\section{Utilizando seL4 como hipervisor}
El siguiente modelo se centra en la lograr la comunicación entre dos VM, las cuales se encuentran en instancias independientes de seL4 funcionando como hipervisor. Este modelo tiene como objetivo validar el \textit{passthrough} de hardware, en este caso, de una interfaz de red entre seL4 y el \textit{guest} Linux. En la figura \ref{diag:esquema_passthrough} se esquematiza la topología del modelo.

\begin{figure}[h!]
    \centering
    \includegraphics[width=0.5\textwidth]{example-image}
    \caption{Dos VMs con un passthrough, ping 2VM.}
    \label{diag:esquema_passthrough}
\end{figure}

Un paso importante de este desarrollo es validar la compatibilidad de las imágenes de Linux obtenidas previamente con el hipervisor seL4. Para ello, en este modelo se utiliza el \textit{output} del primer modelo, la imagen de Linux con soporte para WireGuard.

Como \textit{output} de este modelo se tiene la configuración adecuada para realizar el \textit{passthrough} de un dispositivo de red al Linux \textit{guest}. Además de correcciones que se hayan hecho sobre la imagen de Linux para lograr la compatibilidad con el VMM de seL4.

\section{Implementando el encriptador en seL4}


\begin{itemize}
    \item 4 instancias de QEMU: 2 PCs y 2 encriptadores.
    \item Descripción de zmq\_samples. Virtual Switch. ZeroMQ. Esquemas. 
    \item Problema de usar dos placas de red con una misma IRQ.
\end{itemize}

\noindent\rule{\textwidth}{0.4pt}
%%%%%%%%%%%%%%%%%%%%%%%%%%%%%%%%%%%%%%%%%%%%%%%%%%%%%%%%%%%%%%





\section{Resumen?}
Tabla: modelo, nivel de complejidad, validaciones


\chapter{Implementación en entorno virtualizado} % sistema completo seL4 en QEMU
\section{Diseño del experimento}
\begin{itemize}
    \item Se realizará en completamente en host. ¿Por qué? Simplicidad de implementación.
\end{itemize}
\section{Procedimiento} % Proceso de: linux_QEMU-linux_QEMU -> 3VMsel4_QEMU. 3VMsel4_QEMU
  
\subsection{Construcción de un kernel Linux con soporte para WireGuard}
Actualmente, el VMM de seL4 implementado en CAmkES para la arquitectura x86 cuenta con soporte probado para el kernel Linux 4.9. En este trabajo se utilizó una versión limpia de este kernel, obtenida de la rama \textit{stable} de Linux 4.9.y, en particular la versión 4.9.337, que se encuentra disponible en el repositorio oficial de Linux. Al momento de configurar la compilación del kernel, se tuvo en cuenta la compatibilidad con el hipervisor seL4, cuyos requisitos se encuentran documentados en la página oficial del proyecto \textit{camkes-vm} para seL4 \cite{camkes_vm}.

El soporte nativo para WireGuard fue añadido al kernel Linux en la versión 5.6, sin embargo, el repositorio \textit{wireguard-linux-compat} proporciona un parche que permite compilar WireGuard en versiones anteriores del kernel, como la utilizada \cite{wireguard-compilation}.

\begin{lstlisting}[language=bash, caption={Parche wireguard-linux-compat para kernel 4.9}]
git clone https://git.zx2c4.com/wireguard-linux-compat
./wireguard-linux-compat/kernel-tree-scripts/jury-rig.sh ./linux-stable
\end{lstlisting}

\begin{itemize}
    \item De www.wireguard.com/compilation/kernel-requirements se obtuvieron los requerimientos.
    \item wireguard-linux-compat patch
    \item seL4 soporta kernel 4.9[ref], se compiló este kernel con una adaptación de .config file original de los ejemplos camkes-vm
    \item Buscar referencias para justificar todo. \cite{Laricch2009AAMOP_ICBp3}

    \item Drivers de red compatibles con el hardware utilizado.
\end{itemize}

\subsection{Generación de una imagen de sistema mediante Buildroot}
\begin{itemize}
    \item Se utilizó la versión 2023.02.1 de Buildroot utilizando el kernel modificado.
    \item Se configuró el sistema de archivos para que contenga los binarios necesarios para el funcionamiento de WireGuard y las herramientas de red.
\end{itemize}

\subsection{Adaptación del ejemplo \textit{zmq\_samples}}
\begin{itemize}
    \item Se adaptó el ejemplo de CAmkES para que funcione con el nuevo kernel y la imagen de sistema generada. Basicamente tocar el CMakeLists.txt
    \item ZeroMQ. Problema con iperf3 solucionado aumentando tamaño de buffers.
    \item Incrementar RAM de las VMs.
\end{itemize}

\subsection{Configuración del \textit{passthrough} de interfaz Ethernet}
\begin{itemize}
    \item Interfaz e1000 QEMU. PCI. BARS. IRQ.
    \item Modificar la configuración de la VMM camkes en seL4 para permitir el acceso a los recursos PCI correspondientes.
    \item Solución al problema de utilizar dos interfaces de red. Diferentes IRQ.
    \item Funcionamiento de passthrough (colas).
\end{itemize}

\section{Integración}
\begin{itemize}
    \item Bridge en host. 4 instancias de QEMU. 2 PCs y 2 encriptadores.
\end{itemize}

\section{Validación} %verificar requerimientos


\chapter{Implementación en hardware}
    \section{SuperMicro SYS-E300-9D}
    \begin{itemize}
        \item Supermicro server sys-e300-9d with X11SDV-4C-TLN2F motherboard and intel xeon D-2123IT. Fotito del equipo.
        \item IPMI, COM1, SoL.
    \end{itemize}
    \section{Diseño del experimento}
    \begin{itemize}
        \item Que cosas esperamos validar. Rendimiento y que más?
        \item SetUP.
    \end{itemize}
    \section{Procedimiento}
    \begin{itemize}
        \item 1. Configurar redirección consola serie.
        \item 2. Bootear el Linux para obtener los parámetros para configurar el zmq\_samples (lspci).
        \item 
    \end{itemize}

\chapter{Implementación en entorno virtualizado} % sistema completo seL4 en QEMU

\section{Diseño del experimento}
\begin{itemize}
    \item Se realizará en completamente en host. ¿Por qué? Simplicidad de implementación.
\end{itemize}
\section{Procedimiento} % Proceso de: linux_QEMU-linux_QEMU -> 3VMsel4_QEMU. 3VMsel4_QEMU
  
\subsection{Construcción de un kernel Linux con soporte para WireGuard}
\begin{itemize}
    \item De www.wireguard.com/compilation/kernel-requirements se obtuvieron los requerimientos.
    \item wireguard-linux-compat patch
    \item seL4 soporta kernel 4.9[ref], se compiló este kernel con una adaptación de .config file original de los ejemplos camkes-vm
    \item Buscar referencias para justificar todo. \cite{Laricch2009AAMOP_ICBp3}

    \item Drivers de red compatibles con el hardware utilizado.
\end{itemize}

\subsection{Generación de una imagen de sistema mediante Buildroot}
\begin{itemize}
    \item Se utilizó la versión 2023.02.1 de Buildroot utilizando el kernel modificado.
    \item Se configuró el sistema de archivos para que contenga los binarios necesarios para el funcionamiento de WireGuard y las herramientas de red.
\end{itemize}

\subsection{Adaptación del ejemplo \textit{zmq\_samples}}
\begin{itemize}
    \item Se adaptó el ejemplo de CAmkES para que funcione con el nuevo kernel y la imagen de sistema generada. Basicamente tocar el CMakeLists.txt
    \item ZeroMQ. Problema con iperf3 solucionado aumentando tamaño de buffers.
    \item Incrementar RAM de las VMs.
\end{itemize}

\subsection{Configuración del \textit{passthrough} de interfaz Ethernet}
\begin{itemize}
    \item Interfaz e1000 QEMU. PCI. BARS. IRQ.
    \item Modificar la configuración de la VMM camkes en seL4 para permitir el acceso a los recursos PCI correspondientes.
    \item Solución al problema de utilizar dos interfaces de red. Diferentes IRQ.
    \item Funcionamiento de passthrough (colas).
\end{itemize}

\section{Integración}
\begin{itemize}
    \item Bridge en host. 4 instancias de QEMU. 2 PCs y 2 encriptadores.
\end{itemize}


\section{Validación} %verificar requerimientos

\chapter{Implementación en hardware}
    \section{SuperMicro SYS-E300-9D}
    \begin{itemize}
        \item Supermicro server sys-e300-9d with X11SDV-4C-TLN2F motherboard and intel xeon D-2123IT. Fotito del equipo.
        \item IPMI, COM1, SoL.
    \end{itemize}
    \section{Diseño del experimento}
    \begin{itemize}
        \item Que cosas esperamos validar. Rendimiento y que más?
        \item SetUP.
    \end{itemize}
    \section{Procedimiento}
    \begin{itemize}
        \item 1. Configurar redirección consola serie.
        \item 2. Bootear el Linux para obtener los parámetros para configurar el zmq\_samples (lspci).
        \item 
    \end{itemize}


%%%%%%%%%%%%

% \appendix
% \include{Capitulos/Ap_XX}

% \begin{biblio}
% \bibliography{mibib}
% \end{biblio}

% \begin{postliminary}

% \begin{seccion}{Publicaciones asociadas}
%   \begin{enumerate}
%   \item Mi primer aviso en la revista \textbf{ABC}, 1996
%   \item Mi segunda publicaci\'{o}n en la revista \textbf{ABC}, 1997
%   \end{enumerate}
% \end{seccion}

% \begin{seccion}{Agradecimientos}
% A todos los que se lo merecen, por merecerlo
% \end{seccion}

% \end{postliminary}

%%%%%%%%%

\end{document}

